\section{Special review issue}\label{special-review-issue}

Since Charles Babbage completed his difference engine in 1844, which was
shortly followed by his analytical engine, the civilised world rapidly
underwent a mechanical revolution. Relieving the English Gentleman of
the tedious calculations which accompany industrial endeavours, man has
been given the freedom to expend intellectual capacity on more lofty
endeavours. Through this, Britain has not only flourished herself, but
also shared her great strides with all of the Empire. In this issue we
continue the review of the recent industrial consequences of our great
calculating engines.

Every industry has been affected for the better by our knowledge
machinery, and while our readers will be most familiar with the military
applications of thinking engines --- especially that of cryptography ---
we would like to focus attention on the professions of machine making
and craftsmen. The demand for computing machines has lead to an
exponential increase in the demand for fine-precision gears and
components. Nowhere are gears better manufactured than in Britain, and
our nation will soon have the ability to mass produce near-identical
copies of the series GBG gearsets. It has been said that the Empire is
built on cotton, this newspaper posits that it is kept running by
precision gears.

\section{Historic meeting coming to
Maidstone}\label{historic-meeting-coming-to-maidstone}

Doom prophets, naysayers, and Luddites would have us believe that our
economy will lead to world-wide war on an industrial scale, but the
English Gentleman continues to enjoy great quality of life. This
newspaper has learned that there will be an unofficial meeting between
the cardinal minds behind our economy on DATEDATE at 4 Thornhill Mews
Cross Street, Maidstone, ME14 2SP. The meeting will no doubt further
build Anglo-Franco relationships and set a course for the next
industrial leap. We have compiled biographies of all the members who
shall be present.

\textbf{Wilbur Haborlian}\\
The man whose influence on the industrial age has been second only to
that of the great engine-maker, Charles Babbage. Being the nation's
finest gear maker, Wilbur has not only near-perfected the precision and
tolerance, but he is said to be working on a mass-production system that
will churn out gears by the thousands. The holder of 80 patents, Wilbur
hails from humble beginnings and little is known of his upbringing in
various foster families. Initially rejected by the Royal Society due to
lack of educational pedigree, his techniques and successes speak for
themselves. Thanks to discovery and patronage from the Ryan family,
Wilbur was able to achieve great things. Despite all the yellow-press
rumours of adulterous nature, Wilbur has been married all his life to
his wife, lady Rae Haborlian.

\textbf{Odis Urquart} (Male Luddite)\\
\textbf{Margaret Urquart} (Female Luddite)\\
Self-described old-money Gentleman, Odis is the most known as a member
the now-defunct House of Lords. Not one to shy away from public life, he
acts as the unofficial leader of the Luddites, who continue to preach
their anti-engineering, moral and religious message. While mostly
regarded as harmless, the Luddites still control substantial wealth as a
group, and their nationalistic propaganda is thought by many to be
populist pandering; the Luddites have seen their wealth multiplied many
times by our Industrial Revolution. Hence, they are still consulted on
important matters of economy and state, and Odis is sure to be seen at
any meetings of importance.

\textbf{Sir Doctor Conrad Borthwell} (Male PM)\\
\textbf{Sir Doctor Beula Borthwell} (Female PM)\\
His excellency the Prime Minister and Chair of the Royal Society of
Engineering. Cambridge educated, Sir Doctor Conrad successfully lead
Britain through the transition to constitutional technocracy. While
still derided for his inability to identify and develop Wilbur
Haborlian, one must admit the Prime Minister has had far more successes
than failures in his tenure. Inventor of the fighting system used by the
British Armed forces, published in his text ``Scientific Bayonetting, a
Field Guide to Biological Deconstruction''.

\textbf{Hazel Ryan} (Female Narco heir)\\
\textbf{Dominique Ryan} (Male Narco heir)\\
The the heir to the mighty Ryan fortune; a multinational umbrella
corporation that owns companies in nearly every field. It has been
speculated that the deceased Ryan patriarch, George Alexander Ryan,
amassed his fortune by being the biggest Opium smuggler in Europe, this
has never been proven. The family have expanded to more lucrative
industries in recent years, like industrial warfare equipment.

\textbf{Mortimer Sicario} (Male Narco Sicario)\\
\textbf{Odelia Sicario} (Female Narco Sicario)\\
Mortimer acted as right hand man and chief security adviser to the Ryan
family, and has been with them for 25 years. Being near the Ryan family
necessitates mystery and rumour, and the yellow press speculates that
Mortimer used to be the Ryan hitman. Far be it from this newspaper to
repeat such allegations, and we merely note that anywhere Hazerl Ryan
appears in public, Mortimer is bound to be there.

\textbf{Arnaud Lepetit} (Male French savant)\\
\textbf{Sophie Lepetit} (Female French savant)\\
Representing the newly-independent France, Arnaud is a retired french
spy and intelligence official, who has seen his fair share of active
duty in some of the recent European skirmishes. Now interested only in
diplomacy, he has brokered many deals between the French and other
European interests; ensuring that France has a seat at the European
table.

\textbf{Rae Haborlian} (Female Spouse)\\
Willbur's Cambridge educated wife, she built a chess-playing engine when
she was only fourteen years old. She abandoned a promising career in the
engine-making arts behind to live in Wilbur's shadow as British customs
only allow for one genius per household. She has stayed with her husband
and supported him through the years of poverty, as well as the
succeeding years filled with scandalous rumours implying adultery.

\textbf{Lorraine Stoker} (Female Assistant)\\
Assistant and protégée to Wilbur, this prodigy arose from humble
beginnings to secure multiple scholarships and finally ended up as
permanent researcher on many of Wilbur's projects, where she has been
diligently documenting and publishing his work as quick as he could
produce it. Said to have been orphaned during the drug turf-wars of the
London slums, she received shelter in a Luddite-funded orphanage in
North London where she develop a liking in the engineering arts, despite
their protests and her training in the classics.

\textbf{Earle Welter} (Male CEO)\\
\textbf{Jane Welter} (Female CEO)\\
Chief Executive Officer of British Trans-continental Defence (BTD) Ltd.
BTD Ltd. is partly owned by the Ryan family, and specialises in copper
communications cables and defence communications technology. Earle has
grown the company eight-fold during his tenure, but they have been
accused of selling to both sides of the Anglo-Franco aggression. Whoever
loses in a hypothetical war between these two industrial superpowers,
BTD Ltd. is sure to have a winning balance sheet.

\textbf{Retired Chief Inspector Mycroft Windsor} (Male inspector)\\
\textbf{Retired Chief Inspector Verla Windsor} (Female inspector)\\
Negotiations consultant and mediator, used to work for Scotland Yard. A
self-proclaimed pacifist, he is widely respected and regular attendee at
any events where national and international matters are discussed and
negotiated and is said by many to be the sole proprietor of peace during
recent times where nations have often been at each other's throats.

\section{Letters to the editor}\label{letters-to-the-editor}

Dear Sir

\emph{This newspaper is quick to point out the benefits of the Godless
march towards Mechanisation, but it ignores the good British people who
are either being replaced by machines, or the scourge filtering through
our borders on the beastly, newfangled steel passenger ships with their
trailing black plume. Very soon the British way of life will be
destroyed, and nothing will remain but bastardised remnants of the
classes that once formed the bastion of our great Empire.}

Yours, Luddite for Life

Dear Sir

\emph{In response to the article extolling the virtues of our
Mechanically advanced society --- which received criticisms from our
Luddite brethren as usual --- I would like to state my support for your
thesis that Britons are better off. Take, for example, the recent
advances in the manufacturing of soap. Thanks to advanced detergents,
the average British household is spending considerably less time
scrubbing their their pots, their pans, and their pantaloons. This fact
is quickly lost on the Luddites, whose classes had these duties
performed by servants.}

Yours faithfully, John Q. Public.
